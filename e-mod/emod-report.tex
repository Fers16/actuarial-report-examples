\documentclass[12pt,titlepage]{article}\usepackage[]{graphicx}\usepackage[]{color}
%% maxwidth is the original width if it is less than linewidth
%% otherwise use linewidth (to make sure the graphics do not exceed the margin)
\makeatletter
\def\maxwidth{ %
  \ifdim\Gin@nat@width>\linewidth
    \linewidth
  \else
    \Gin@nat@width
  \fi
}
\makeatother

\definecolor{fgcolor}{rgb}{0.345, 0.345, 0.345}
\newcommand{\hlnum}[1]{\textcolor[rgb]{0.686,0.059,0.569}{#1}}%
\newcommand{\hlstr}[1]{\textcolor[rgb]{0.192,0.494,0.8}{#1}}%
\newcommand{\hlcom}[1]{\textcolor[rgb]{0.678,0.584,0.686}{\textit{#1}}}%
\newcommand{\hlopt}[1]{\textcolor[rgb]{0,0,0}{#1}}%
\newcommand{\hlstd}[1]{\textcolor[rgb]{0.345,0.345,0.345}{#1}}%
\newcommand{\hlkwa}[1]{\textcolor[rgb]{0.161,0.373,0.58}{\textbf{#1}}}%
\newcommand{\hlkwb}[1]{\textcolor[rgb]{0.69,0.353,0.396}{#1}}%
\newcommand{\hlkwc}[1]{\textcolor[rgb]{0.333,0.667,0.333}{#1}}%
\newcommand{\hlkwd}[1]{\textcolor[rgb]{0.737,0.353,0.396}{\textbf{#1}}}%

\usepackage{framed}
\makeatletter
\newenvironment{kframe}{%
 \def\at@end@of@kframe{}%
 \ifinner\ifhmode%
  \def\at@end@of@kframe{\end{minipage}}%
  \begin{minipage}{\columnwidth}%
 \fi\fi%
 \def\FrameCommand##1{\hskip\@totalleftmargin \hskip-\fboxsep
 \colorbox{shadecolor}{##1}\hskip-\fboxsep
     % There is no \\@totalrightmargin, so:
     \hskip-\linewidth \hskip-\@totalleftmargin \hskip\columnwidth}%
 \MakeFramed {\advance\hsize-\width
   \@totalleftmargin\z@ \linewidth\hsize
   \@setminipage}}%
 {\par\unskip\endMakeFramed%
 \at@end@of@kframe}
\makeatother

\definecolor{shadecolor}{rgb}{.97, .97, .97}
\definecolor{messagecolor}{rgb}{0, 0, 0}
\definecolor{warningcolor}{rgb}{1, 0, 1}
\definecolor{errorcolor}{rgb}{1, 0, 0}
\newenvironment{knitrout}{}{} % an empty environment to be redefined in TeX

\usepackage{alltt}
\usepackage{longtable, booktabs, caption, hyperref}

\definecolor{Blue}{rgb}{0,0,0.8}
\hypersetup{%
colorlinks,%
plainpages=true,%
linkcolor=black,%
citecolor=black,%
urlcolor=Blue,%
%pdfstartview=FitH,% or Fit
pdfstartview={XYZ null null 1},%
pdfview={XYZ null null null},%
pdfpagemode=UseNone,% for no outline
pdfauthor={Friedrich Leisch and R-core},%
pdftitle={Sweave User Manual},%
pdfsubject={R vignette documentation system}%
}
\IfFileExists{upquote.sty}{\usepackage{upquote}}{}
\begin{document}





\title{Example Member \\
       North Carolina \\
       Workers' Compensation Emod Report}
\author{Prepared by Ractuary}
\date{\today}

\maketitle

\begin{knitrout}
\definecolor{shadecolor}{rgb}{0.969, 0.969, 0.969}\color{fgcolor}\begin{kframe}


{\ttfamily\noindent\color{warningcolor}{\#\# Warning in left\_join\_impl(x, y, by\$x, by\$y): joining factors with different levels, coercing to character vector}}\end{kframe}
\end{knitrout}

\section{Background}

The workers compensation emod calculation is based off a the review of loss experience over the 3 most recent fully earned policy years ending 6 months prior to the incurred loss \& ALAE evaluation date.

The calculation depends on weighting and exposure factors as determined by the \href{http://www.ncrb.org/ncrb/AboutNCRB/tabid/55/Default.aspx}{North Carolina Rating Bureau} (NCRB) statistical review of the workers' compensation insurance market.

\subsection{Emod Formula}

$\frac{A_p+w*A_e+(1-E_e)+b}{E_p+E_e+b}$ \\

Where:
\begin{itemize}
\item All claims consisting of only medical and expense incurred loss \& ALAE are reduced by 70\%
\item $A$ are the actual incurred loss \& ALAE
\item $E$ are the exxpected incurred loss \& ALAE
\item The $_p$ subscript indicates primary\footnote{primary loss \& ALAE are all losses below a certain dollar value determined by NCRB} incurred loss \& ALAE
\item The $_e$ subscript indicate excess\footnote{excess loss \& ALAE are all losses above the dollar value used to determine primary loss \& ALAE}
\item w is a weighting factor dependent on $E$
\item b is weighting value  dependent on $E$
\end{itemize}

\pagebreak

\section{Example Member Emod}


% all content after this command has new numbering system
\appendix

\end{document}
